% (This is included by thesis.tex; you do not latex it by itself.)

\begin{abstract}

% The text of the abstract goes here.  If you need to use a \section
% command you will need to use \section*, \subsection*, etc. so that
% you don't get any numbering.  You probably won't be using any of
% these commands in the abstract anyway.
As the processor clock frequency stopped growing, there has been
a move towards parallel computers and more recently, heterogeneous
computing platforms.
As high level synthesis (HLS) moves towards mainstream
adoption among FPGA designers, it has proven to be
an effective method for rapid hardware generation. However,
in the context of offloading compute intensive software kernels
to FPGA accelerators, current HLS tools do not always take
full advantage of the hardware platforms. In this paper, we
present an automatic flow to refactor and restructure processorcentric
software implementations, making them better suited
for FPGA platforms. The methodology generates pipelines that
decouple memory operations and data access from computation.
The resulting pipelines have much better throughput due to their
efficient use of the memory bandwidth and improved tolerance
to data access latency. The methodology complements existing
work in high-level synthesis, easing the creation of heterogeneous
systems with high performance accelerators and general purpose
processors. With this approach, for a set of non-regular algorithm
kernels written in C, a performance improvement of 3.3 to 9.1x
is observed over direct C-to-Hardware mapping using a state-ofthe-art
HLS tool.

\end{abstract}
